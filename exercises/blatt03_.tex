%!TEX TS-program = pdflatex
%!TEX TS-options = -shell-escape
% % % % %   Die folgenden Zeilen müssen ihre Zeilennummern 4 und 5 behalten !!!    % % % % %
\newcommand{\printpraesenzlsg}{false}
\newcommand{\printloesungen}{true}
\newcommand{\printbewertungen}{false}
% % % % %   \newcommand{\printloesungen}{false}                                    % % % % %
\newcommand{\blattnummer}{3}
%\newcommand{\abgabetermin}{\textcolor{red}{bis 11.04.2022, 09:00 Uhr}}
\input{include/config.tex}

% Änderungen 2020: Hinweise auf Digitallehre angepasst; Rechnergeschichte entfernt; Blatt 1 und 2 zusammengefasst.
\begin{document}
\iforiginal{\input{include/kopf.tex}}

% \begin{notes} \small
% 	\textbf{Abgabetermine für Blatt 1:}
	
% 	Aufgaben 1.2/1.3: Montag, 11. April, 09:00 Uhr \\
% 	Aufgabe 1.4: Mittwoch, 20. April, 18:00 Uhr
% \end{notes}

\aufgabetitel{$5$}{Mengen}
Bestimmen Sie die folgenden Mengen:
\begin{enumerate}[(i)]
  \item $(\{1,2\}\times\{3,4\})\cup\{1,2,3\}$
  \item $\{a,b\}\times\mathscr{P}(\{1,2\})$
  \item $\mathscr{P}(\{1,2\})\cap\mathscr{P}(\{1\})$
\end{enumerate}

\begin{loesung}
\begin{enumerate}[(i)]
  \item $\{1,2,3,(1,3),(1,4),(2,3),(2,4)\}$
  \item $\{(a,\emptyset),(b,\emptyset),(a,\{1\}),(b,\{1\}),(a,\{2\}),(b,\{2\}),(a,\{1,2\}),(b,\{1,2\})\}$
  \item $\{\emptyset, \{1\}\}$
  
\end{enumerate}
\end{loesung}

\aufgabetitel{$4$}{Mengenbeweis}\\
Zeigen Sie für beliebige Mengen $A,B$:

  \[A\subseteq B\Leftrightarrow \mathscr{P}(A)\subseteq\mathscr{P}(B)\]


\begin{loesung}
  \glqq$\Rightarrow$\grqq:

  Sei $M\in\mathscr{P}(A)$. Wir wollen zeigen, dass dann auch $M\in\mathscr{P}(B)$.
  
  Dann ist $M\subseteq A$. Da aber $A\subseteq B$ ist auch $M\subseteq B$ und damit $M\in\mathscr{P}(B)$.

  \glqq$\Leftarrow$\grqq:

  Sei $x\in A$. Wir wollen zeigen, dass dann $x\in B$. Es ist $\{x\}\subseteq A$, also $\{x\}\in\mathscr{P}(A)$.
  
  Wegen Voraussetzung ist aber $\mathscr{P}(A)\subseteq\mathscr{P}(B)$, also $\{x\}\in\mathscr{P}(B)$ und damit $x\in B$. 

  \qed
\end{loesung}


\end{document}
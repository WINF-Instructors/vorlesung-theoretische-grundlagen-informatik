%!TEX TS-program = pdflatex
%!TEX TS-options = -shell-escape
% % % % %   Die folgenden Zeilen müssen ihre Zeilennummern 4 und 5 behalten !!!    % % % % %
\newcommand{\printpraesenzlsg}{false}
\newcommand{\printloesungen}{false}
\newcommand{\printbewertungen}{false}
% % % % %   \newcommand{\printloesungen}{false}                                    % % % % %
\newcommand{\blattnummer}{2}
%\newcommand{\abgabetermin}{\textcolor{red}{bis 11.04.2022, 09:00 Uhr}}
\input{include/config.tex}

% Änderungen 2020: Hinweise auf Digitallehre angepasst; Rechnergeschichte entfernt; Blatt 1 und 2 zusammengefasst.
\begin{document}
\iforiginal{\input{include/kopf.tex}}

% \begin{notes} \small
% 	\textbf{Abgabetermine für Blatt 1:}
	
% 	Aufgaben 1.2/1.3: Montag, 11. April, 09:00 Uhr \\
% 	Aufgabe 1.4: Mittwoch, 20. April, 18:00 Uhr
% \end{notes}

\aufgabetitel{$5$}{Mengen} \\
Bestimmen Sie die folgenden Mengen:
\begin{enumerate}[(i)]
  \item $(\{1,2\}\times\{3,4\})\cup\{1,2,3\}$
  \item $\mathscr{P}(\{1,2,3\})\backslash\mathscr{P}(\{1,2\})$
  \item $\bigcap_{i\in\{2,6\}}\{\frac{i}{2},i+1\}$ 
  \item $\bigcup_{n\in\mathbb{N}}\{n,n+1,2n\}$
  \item $\mathscr{P}(\emptyset)$
\end{enumerate}


\aufgabetitel{$5$}{Mengenbeweise} \\
Beweisen Sie folgende Aussagen:
\begin{enumerate}[(i)]
  \item $A\subseteq B\cap C\Leftrightarrow A\subseteq B\wedge A\subseteq C$
  \item $A\backslash(B\cup C)=(A\backslash B)\cap (A\backslash C)$
  \item $\bigcap_{n\in\mathbb{N}}\{m\in\mathbb{N}|m\geq n\}=\emptyset$
  \item $\left(\bigcup_{i\in I}D_i\right)\cap B=\bigcup_{i\in I}(D_i\cap B)$
  \item $\bigcap_{\varepsilon\in\mathbb{R}\backslash\{0\}}\{x\in\mathbb{R}||x-\pi|\leq |\varepsilon|\}=\{\pi\}$
\end{enumerate}

\aufgabetitel{$4$}{Symmetrische Differenz}\\
  Unter 
  \[A\triangle B:=(A\backslash B)\cup (B\backslash A)\] versteht man die \textit{symmetrische Differenz} der Mengen $A$ und $B$.
  \begin{enumerate}[(i)]
    \item Machen Sie sich anhand eines Venn-Diagramms klar, was unter der symmetrischen Differenz anschaulich zu verstehen ist.
    \item Beweisen Sie: $\forall A,B:A\triangle B=(A\cup B)\backslash(A\cap B)$.
  \end{enumerate}
\aufgabetitel{$2$}{Beweisen oder Widerlegen} \\
Beweisen oder widerlegen Sie:\\
Aus $A_1\cap A_2\neq\emptyset, A_2\cap A_3\neq\emptyset$ und $A_1\cap A_3\neq\emptyset$ folgt $\bigcap_{i\in\{1.2.3\}}A_i\neq\emptyset$.


\end{document}
%!TEX TS-program = pdflatex
%!TEX TS-options = -shell-escape
% % % % %   Die folgenden Zeilen müssen ihre Zeilennummern 4 und 5 behalten !!!    % % % % %
\newcommand{\printpraesenzlsg}{false}
\newcommand{\printloesungen}{false}
\newcommand{\printbewertungen}{false}
% % % % %   \newcommand{\printloesungen}{false}                                    % % % % %
\newcommand{\blattnummer}{1}
%\newcommand{\abgabetermin}{\textcolor{red}{bis 11.04.2022, 09:00 Uhr}}
\input{include/config.tex}

% Änderungen 2020: Hinweise auf Digitallehre angepasst; Rechnergeschichte entfernt; Blatt 1 und 2 zusammengefasst.
\begin{document}

\iforiginal{\input{include/kopf.tex}}
\aufgabetitel{$6$}{Mengen und Funktionen}\\
Gegeben seien die Mengen $A=\{\{a,b,c\},d\}$ und $B=\{a,d\}$.
\begin{enumerate}[a)]
\item Geben Sie die Menge $A\cup B$ an
\item Geben Sie die Menge $A\cap B$ an
\item Geben Sie die Menge $A\setminus B$ an.

\end{enumerate}

\aufgabetitel{$10$}{Potenzmengen}\\
\begin{enumerate}[a)]
    \item Geben Sie die Potenzmenge $\mathscr{P}(\{1,2\})$ an.
    \item Geben Sie die Potenzmenge $\mathscr{P}(\mathscr{P}(\emptyset))$ an.
    \item Ist $\emptyset\in\mathscr{P}(\emptyset)$? Begründen Sie Ihre Antwort. 
    \item Ist $\emptyset\subseteq\mathscr{P}(\emptyset)$? Begründen Sie Ihre Antwort.
    \item Ist $1\in\mathscr{P}(\{1,2\})$? Begründen Sie Ihre Antwort.
    \item Ist $1\subseteq\mathscr{P}(\{1,2\})$? Begründen Sie Ihre Antwort.
\end{enumerate}

\aufgabetitel{$10$}{Relationen}\\
Gegeben sei die Menge $A=\{1,2,3\}$. Es sei $R\subseteq A\times A$ mit $R=\{(1,1),(2,2),(3,3),(1,2),(2,1)\}$.
\begin{enumerate}[a)]
    \item Welche Eigenschaften hat die Relation $R$? (zur Auswahl stehen reflexiv, symmetrisch, antisymmetrisch, transitiv). Begründen Sie Ihre Antwort.
    \item Betrachten wir die Relation $S\subseteq \mathscr{P}(A)\times \mathscr{P}(A)$ mit $(X,Y)\in S\Leftrightarrow X\subseteq Y$. Zeichnen Sie die Relation $S$ als Graphen.
    \item Welche Eigenschaften hat die Relation $S$? (zur Auswahl stehen reflexiv, symmetrisch, antisymmetrisch, transitiv). Begründen Sie Ihre Antwort.
    \item Geben Sie ein Hasse-Diagramm der Relation $S$ an.
    \item Was sind die minimalen, kleinsten, maximalen und gr"o{\ss}ten Elemente von in $S$ (falls vorhanden)? Begründen Sie Ihre Antwort.

\end{enumerate}

\aufgabetitel{$6$}{Mengenbeweis}\\
Zeigen Sie, dass für alle Mengen $A,B,C$ gilt: $A\setminus (B\cup C) = (A\setminus B)\cap (A\setminus C)$.

\hspace{5cm}

\underline{\textbf{Formelsammlung}}\\
Hier eine kleine Formelsammlung. Sie ist nicht vollständig, enthält aber alle wichtigen Statements / Definitionen, die man brauchen könnte.
\begin{enumerate}
\item Aussagen- und Pr"adikatenlogik
\begin{enumerate}
    \item Distributivgesetz: $A\wedge (B \vee C) \Leftrightarrow (A\wedge B) \vee (A\wedge C)$
    \item Distributivgesetz: $A\vee (B \wedge C) \Leftrightarrow (A\vee B) \wedge (A\vee C)$
    \item DeMorgan: $\neg(A\wedge B) \Leftrightarrow \neg A \vee \neg B$
    \item DeMorgan: $\neg(A\vee B) \Leftrightarrow \neg A \wedge \neg B$
    \item Idempotenz: $A\wedge A \Leftrightarrow A$
    \item Idempotenz: $A\vee A \Leftrightarrow A$
    \item $A\wedge \neg A \Leftrightarrow \bot$
    \item $A\vee \neg A \Leftrightarrow \top$
    \item $\neg\neg A \Leftrightarrow A$
    \item $\neg\forall x\in M: A(x) \Leftrightarrow \exists x\in M: \neg A(x)$
    \item $\neg\exists x\in M: A(x) \Leftrightarrow \forall x\in M: \neg A(x)$
\end{enumerate}
\item Mengen
\begin{enumerate}
    \item Teilmenge: $A\subseteq B \Leftrightarrow \forall x\in A: x\in B$
    \item Potenzmenge: $\mathscr{P}(A) = \{B\mid B\subseteq A\}$
    \item Vereinigung: $A\cup B = \{x\mid x\in A \vee x\in B\}$
    \item Schnittmenge: $A\cap B = \{x\mid x\in A \wedge x\in B\}$
    \item Differenzmenge: $A\setminus B = \{x\mid x\in A \wedge x\notin B\}$
    \item Distributivgesetz: $A\cap (B \cup C) \Leftrightarrow (A\cap B) \cup (A\cap C)$
    \item Distributivgesetz: $A\cup (B \cap C) \Leftrightarrow (A\cup B) \cap (A\cup C)$
    \item DeMorgan: $A\setminus (B\cup C) \Leftrightarrow (A\setminus B) \cap (A\setminus C)$
    \item DeMorgan: $A\setminus (B\cap C) \Leftrightarrow (A\setminus B) \cup (A\setminus C)$
    \item Es ist $\bigcup_{i\in I}A_i = \{x\mid \exists i\in I: x\in A_i\}$. 
    \item Es ist $\bigcap_{i\in I}A_i = \{x\mid\forall i\in I: x\in A_i\}$.
\end{enumerate}
\item Relationen
\begin{enumerate}
    \item F"ur Mengen $M,N$ ist $R\subseteq M\times N$ eine Relation von $M$ nach $N$.
    \item $R\subseteq M\times M$ ist reflexiv, wenn $\forall x\in M: (x,x)\in R$.
    \item $R\subseteq M\times M$ ist symmetrisch, wenn $\forall x,y\in M: (x,y)\in R \Rightarrow (y,x)\in R$.
    \item $R\subseteq M\times M$ ist antisymmetrisch, wenn $\forall x,y\in M: (x,y)\in R \wedge (y,x)\in R \Rightarrow x=y$.
    \item $R\subseteq M\times M$ ist transitiv, wenn $\forall x,y,z\in M: (x,y)\in R \wedge (y,z)\in R \Rightarrow (x,z)\in R$.
    \item $R\subseteq M\times M$ ist eine "Aquivalenzrelation, wenn $R$ reflexiv, symmetrisch und transitiv ist.
    \item $R\subseteq M\times M$ ist eine Ordnungsrelation, wenn $R$ reflexiv, antisymmetrisch und transitiv ist.
    \item F"ur eine "Aquivalenzrelation $\sim$ auf $M$ ist $[x]=\{y\in M\mid x\sim y\}$ die "Aquivalenzklasse von $x$, $M/\sim = \{[x]\mid x\in M\}$ die Menge der "Aquivalenzklassen oder Quotientenmenge von $M$ modulo $\sim$. Die Menge der "Aquivalenzklassen ist eine Partition von $M$. Umgekehrt induziert jede Partition eine "Aquivalenzrelation.
    \item F"ur eine Ordnungsrelation $\leq$ auf $M$ und $X\subseteq M$ ist $g$ ein kleinstes Element von $X$, wenn $\forall x\in X: g\leq x$, $g$ ein minimales Element von $X$, wenn $\forall g'\in X: g'\leq g\Rightarrow g=g'$, maximale und gr"o{\ss}te Elemente analog.
\end{enumerate}
\item Funktionen
\begin{enumerate}
    \item Eine Funktion $f:X\to Y$ ist eine Relation (also $f\subseteq X\times Y$), die jedem Element aus der Definitionsmenge $X$ genau ein Element aus der Zielmenge $Y$ zuordnet.
    \item $f$ ist injektiv, wenn $\forall x_1,x_2\in X: f(x_1)=f(x_2) \Rightarrow x_1=x_2$.
    \item $f$ ist surjektiv, wenn $\forall y\in Y: \exists x\in X: f(x)=y$.
    \item $f$ ist bijektiv, wenn $f$ injektiv und surjektiv ist.
    \item Die Umkehrfunktion $f^{-1}$ ist definiert $f^{-1}(Y)=\{x\in X\mid\exists y\in Y: y=f(x)\}$
\end{enumerate}
\end{enumerate}
\end{document}
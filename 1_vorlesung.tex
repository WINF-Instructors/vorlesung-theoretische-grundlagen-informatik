\documentclass{beamer}
\usetheme{CambridgeUS}
\usepackage{tikz}
\usepackage[german]{babel}
\usetikzlibrary{matrix,arrows,fit,positioning, mindmap, trees}

\usepackage[latin1]{inputenc}
\usefonttheme{professionalfonts}
\usepackage{times}
\usepackage{xmpmulti}
\usepackage{animate}
\usepackage{amsmath}
\usepackage{verbatim}
\usepackage{mathrsfs}
%\usetheme{Boadilla}
%\usecolortheme{crane}
\title[Mathematik I]{Mathematik I: \\Theoretische Grundlagen der Informatik}


\author{Prof.\ Dr.-Ing. Sebastian Schlesinger}
%\institute{Berlin School of Economics and Law}
\date{\today}
\begin{document}
 \begin{frame}
\titlepage
\end{frame}

%\begin{frame}
% \frametitle{Outline}
% \tableofcontents
% \end{frame}
%\section{Notationen}
%\subsection{Symmetric va. asymmetric ciphers}
 
\begin{frame}
\frametitle{Aussagen}

Unter einer \textbf{Aussage} versteht man einen sprachlichen Ausdruck, dem man eindeutig einen der beiden Wahrheitswerte w (\glqq wahr\grqq) bzw. f (\glqq falsch\grqq) zuordnen kann. 

Aussagen werden mit Gro\ss buchstaben bezeichnet, \[A:Beschreibung\]
und k"onnen mit logischen Operationen verkn"upft werden. Grundlegende mathematische Aussagen, die nicht aus anderen Aussagen abgeleitet werden k"onnen, nennt man \textbf{Axiome}.
\end{frame}

\begin{frame}
  \frametitle{Beispiele von Aussagen}
  \begin{itemize}
    \item Wahre Aussage A: Jede nat"urliche Zahl ist ein Produkt von Primzahlen.
    \item Falsche Aussage B: Jede Primzahl ist ungerade
    \item Unbewiesene Vermutung (wahr oder falsch, d.h. eine Aussage, bei der der Wahrheitswert noch nicht entschieden werden konnte) C: Es gibt unendlich viele Primzahlzwillinge.
    \item Keine Aussage (Feststellung ohne Wahrheitswert) D: Freitag der dreizehnte ist ein Ungl"uckstag.
  \end{itemize}
      
\end{frame}

\begin{frame}
  \frametitle{Logische Operationen}
  Logische Aussagen k"onnen durch die in der folgenden Tabelle angegebenen Operationen verkn"upft werden.
  
  \begin{tabular}[h]{c|cc|c}
   
    Bezeichnung & Schreibweise & (Sprechweise) & wahr, gdw \\
    \hline
    Negation & $\neg A$ & (nicht A) & A falsch ist \\
    Konjunktion & $A\wedge B$ & (A und B) & A und B wahr sind \\
    Disjunktion & $A\vee B$ & (A oder B) & A oder B wahr ist \\
    Implikation & $A\Rightarrow B$ & (wenn A dann B) & A falsch oder B wahr \\
    "Aquivalenz & $A\Leftrightarrow B$ & (A "aquivalent B) & A und B "aquivalent 


    \end{tabular}
\end{frame}

\begin{frame}
  \frametitle{Bindungsst"arke}
  Um in logischen Ausdr"ucken Klammern zu sparen, wird festgelegt, dass $\neg$ st"arker bindet als $\wedge$ sowie $\vee$ und diese wiederum st"arker als $\Rightarrow, \Leftrightarrow$.
\end{frame}

\begin{frame}
  \frametitle{Wahrheitstabelle}
  In der folgenden Tabelle sind die Wahrheitswerte der vorgestellten Verkn"upfungen angegeben. Dabei steht w f"ur wahr und f f"ur falsch.
  \begin{tabular}[h]{c|c|c|c|c|c|c}
   
    $A$ & $B$ & $\neg A$ & $A\wedge B$ & $A\vee B$ & $A\Rightarrow B$ & $A\Leftrightarrow B$ \\
    \hline
    w & w & f & w & w & w & w \\
    w & f & f & f & w & f & f \\
    f & w & w & f & w & w & f \\
    f & f & w & f & f & w & w
    
  \end{tabular}
  
  \vspace{5mm}
  \textbf{Hinweis:} Statt $w$ f"ur \textit{wahr} und $f$ f"ur \textit{falsch} werden auch die Symbole $\top$ und $\bot$ verwendet.
  
\end{frame}

\begin{frame}
  \frametitle{Gesetze f"ur logische Operationen}
  F"ur logische Operationen gelten die folgenden Identit"aten. 
  \begin{itemize}
    \item Assoziativgesetze: \[(A\wedge B)\wedge C = A\wedge(B\wedge C)\] \[(A\vee B)\vee C = A\vee(B\vee C)\]
    \item Kommutativgesetze: \[A\wedge B = B\wedge A\]\[A\vee B = B\vee A\]
    \item Distributivgesetze: \[A \wedge (B\vee C) = (A\wedge B)\vee (A\wedge C)\] \[A\vee (B\wedge C) = (A\vee B)\wedge (A\vee C)\]
  \end{itemize}
   
\end{frame}


\begin{frame}
  \frametitle{Gesetze f"ur logische Operationen}
  F"ur logische Operationen gelten die folgenden Identit"aten. 
  \begin{itemize}
    \item De Morgansche Regeln: \[\neg(A\wedge B) = (\neg A)\vee(\neg B)\]\[\neg(A\vee B) = (\neg A)\wedge(\neg B)\]
    \item Idempotenz: \[\neg(\neg A) = A\]\[A\vee A = A\]\[A\wedge A = A\]
    \item K"urzungsregeln: \[A\vee\bot=A\]\[A\wedge\top=A\]
  \end{itemize}
   
\end{frame}

\begin{frame}{Pr"adikatenlogik}
  Wir f"uhren nun Quantoren ein. Neben den atomaren Aussagen der Aussagenlogik m"ochte man in der Lage sein, Aussagen zu formulieren, die praktisch von Parametern abh"angen.
  \begin{definition}[Quantoren]
    Wir definieren folgende Notation:
    \begin{itemize}
      \item Mit $\forall x:A(x)$ definieren wir, dass eine Aussage $A$, die eine Variable $x$ beinhaltet, f"ur alle Werte von $x$ gelten soll ("ublicherweise wird dann auch eine Grundmenge, aus denen die $x$ stammen sollen, angegeben).
      \item Mit $\exists x:A(x)$ dr"ucken wir aus, dass es ein $x$ geben soll, so dass $A(x)$ gilt.
    \end{itemize}
  \end{definition}

  Die Aussagen $A(x)$ k"onnen komplexe Aussagen mit Verkn"upfungen sein und es lassen sich auch Quantoren kombinieren und man k"urzt oft ab.
\end{frame}

\begin{frame}{Beweismethoden}
  Wie f"uhrt man nun einen Beweis? Es gibt verschiedene Beweismethoden. Die wichtigsten sind:
  \begin{itemize}
    \item Direkter Beweis: Man beweist direkt $A\Rightarrow B$, also dass $B$ aus der Annahme von $A$ folgt.
    \item Man verwendet die Umkehrung von $A\Rightarrow B$, also $\neg B\Rightarrow\neg A$
    \item Indirekter Beweis: Man nimmt an, $A$ gelte und folgert dann $B$, was aber im Widerspruch zu $A$ ist. Praktisch folgert man also $A\wedge\neq A$, was falsch sein muss. Also muss die Annahme falsch gewesen sein und da sie das Gegenteil von dem ist was man beweisen m"ochte, ist der Beweis komplett.
  \end{itemize}
  
\end{frame}
\begin{frame}
  \frametitle{Mengendefinition}
  \begin{definition}[Naive Mengendefinition]
    Eine Menge ist die Zusammenfassung von bestimmten unterschiedlichen
    Objekten (die Elemente der Menge) zu einem neuen Ganzen.
    Wir schreiben $x\in M$, falls das Objekt $x$ zur Menge $M$ geh"ort.
    Wir schreiben $x\notin M$, falls das Objekt $x$ nicht zur Menge $M$ geh"ort.
    Falls $x\in M$ und $y\in M$ gilt, schreiben wir auch $x, y \in M$.
    Eine Menge, welche nur aus endlich vielen Objekten besteht (eine endliche
    Menge), kann durch explizite Auflistung dieser Elemente spezifiziert
    werden.
  \end{definition}
    Beispiel: $M=\{2,3,5,7\}$.

    Hierbei spielt die Reihenfolge der Auflistung keine Rolle:
    \[\{2,3,5,7\}=\{7,5,3,2\}\]
    Auch Mehrfachauflistungen spielen keine Rolle:
    \[\{2,3,5,7\}=\{2,2,2,3,3,5,7\}\]
  \end{frame}

\begin{frame}{Mengennotation}
Mengen k"onnen definiert werden durch
\begin{itemize}
  \item Aufz"ahlung der Elemente
  \item Formulierung von Bedingungen in der Form $M:=\{x|p(x)\}$, wobei $p$ ein \textit{Pr"adikat}, also eine Aussage ist, die $x$ enth"alt, so dass man jeweils bei Einsetzen von $x$ entscheiden kann, ob sie wahr (und damit das Element zur Menge geh"ort) oder falsch ist (und damit das Element $x$ nicht zu $M$ geh"ort).
\end{itemize}
  Wir schreiben zur Abk"urzung auch $M:=\{x\in N|p(x)\}$ statt $M:=\{x|x\in N\wedge p(x)\}$.
\end{frame}
\begin{frame}
  \frametitle{Besondere Mengen}
  Eine besonders wichtige Menge ist die leere Menge $\emptyset = \{\}$, die keinerlei
Elemente enth"alt.

In der Mathematik hat man es h"aufig auch mit unendlichen Mengen zu
tun (Mengen, die aus unendlich vielen Objekten bestehen).
Solche Mengen k"onnen durch Angabe einer Eigenschaft, welche die
Elemente der Menge auszeichnet, spezifiziert werden.

Beispiele:
\begin{itemize}
  \item $\mathbb{N}=\{0,1,2,3,\dots\}$
  \item $\mathbb{Z}=\{\dots,-2,-1,0,1,2,\dots\}$
  \item $\mathbb{Q}=\{\frac{p}{q}|p,q\in\mathbb{Z}, q\neq 0\}$
  
\end{itemize}
\end{frame}

\begin{frame}{Warum naive Mengenlehre?}
  Die \glqq Definition\grqq der Menge ist anf"allig f"ur Widerspr"uche, z.B. die \textbf{Russelsche Antinomie}:

  Man bilde die Menge aller Mengen, die sich nicht selbst als Element enthalten, in Formeln:
  \[M:=\{N|N\notin N\}\]
  Frage: Gilt $M\in M$? Das f"uhrt auf einen Widerspruch.
  
  Daher hat man die Mengenlehre mit dem \textbf{Zermelo-Fraenkelschen Axiomensystem} auf ein solides Fundament gehoben. Mehr dazu im optionalen Inhalt (ist zu kompliziert f"ur eine erste Einf"uhrung).
\end{frame}

\begin{frame}{Teilmengen}
  \begin{definition}[Teilmenge]
    Seien $A$ und $B$ Mengen. $A\subseteq B$ bedeutet $A$ ist \textit{Teilmenge} von $B$, genau dann, wenn 
    \[\forall x:x\in A\Rightarrow x\in B\] oder "aquivalent dazu \[\forall x\in A:x\in B\]
  \end{definition}
  
  \begin{lemma}[Gleichheit von Mengen]
    Seien $A$ und $B$ Mengen. Es ist $A=B$ genau dann, wenn $\forall x:x\in A\Leftrightarrow x\in B$, was wiederum "aquivalent ist zu
    \[A\subseteq B\wedge B\subseteq A\]  
  \end{lemma}
  Um also die Gleichheit von zwei Mengen zu zeigen beweist man "ublicherweise erst $A\subseteq B$ und dann $B\subseteq A$.
\end{frame}

\begin{frame}{Mengenoperationen}
  \begin{definition}[Mengenoperationen]
    Seien $A$ und $B$ Mengen. Wir definieren:
    \begin{itemize}
      \item $A\cap B:=\{x|x\in A\wedge x\in B\}$, den \textit{Schnitt} von $A$ und $B$
      \item $A\cup B:=\{x|x\in A\vee x\in B\}$, die \textit{Vereinigung} von $A$ und $B$
      \item $A\backslash B:=\{x\in A|x\notin B\}$, die \textit{Differenz} von $A$ und $B$
    \end{itemize}
  \end{definition}
\end{frame}

\begin{frame}{Venn-Diagramme}
  Die Operationen lassen sich in \textbf{Venn-Diagrammen} visualisieren.
  \begin{tikzpicture}
 
    % Set A
    \node [draw,
        circle,
        minimum size =3cm,
        label={135:$A$}] (A) at (0,0){};
     
    % Set B
    \node [draw,
        circle,
        minimum size =3cm,
        label={45:$B$}] (B) at (1.8,0){};
     
    % Set intersection label
    \node at (0.9,0) {$A\cap B$};
     
    \end{tikzpicture}
    \begin{tikzpicture}
 
      % Set A
      \node [circle,
          fill=orange,
          minimum size =3cm,
          label={135:$A$}] (A) at (0,0){};
       
      % Set B
      \node [circle,
          fill=orange,
          minimum size =3cm,
          label={45:$B$}] (B) at (1.8,0){};
       
      % Circles outline
      \draw[white,thick] (0,0) circle(1.5cm);
      \draw[white,thick] (1.8,0) circle(1.5cm);
       
      % Union text label
      \node[orange!90!black] at (0.9,1.8) {$A\cup B$};
       
      \end{tikzpicture}
    
      \begin{tikzpicture}[thick,
        set/.style = { circle, minimum size = 3cm}]
     
    % Set A
    \node[set,fill=orange,label={135:$A$}] (A) at (0,0) {};
     
    % Set B
    \node[set,fill=white,label={45:$B$}] (B) at (0:2) {};
     
    % Circles outline
    \draw (0,0) circle(1.5cm);
    \draw (2,0) circle(1.5cm);
     
    % Difference text label
    \node[left,white] at (A.center){$A-B$};
     
    \end{tikzpicture}
\end{frame}

\begin{frame}{Weitere Mengen und Eigenschaften}
  \begin{definition}[Potenzmenge]
    Mit \[\mathscr{P}(A):=2^A:=\{B|B\subseteq A\}\] bezeichnen wir die \textbf{Potenzmenge}, die Menge aller Teilmengen von $A$.

  \end{definition}
  \begin{definition}[Disjunktheit]
    Zwei Mengen $A$ und $B$ sind \textit{disjunkt}, falls $A\cap B=\emptyset$ gilt.
  \end{definition}
\end{frame}

\begin{frame}{Beispiele f"ur Mengenaussagen}
Es gilt:
\begin{itemize}
  \item $\forall A:\emptyset\subseteq A$
  \item $\forall A:A\subseteq A$
  \item $\mathbb{N}\subseteq\mathbb{Z}\subseteq\mathbb{Q}\subseteq\mathbb{R}$
  \item $\{1,2,3\}\cap\{4,5,6\}=\emptyset$
  \item $\mathscr{P}(\{1,2\})=\{\emptyset,\{1\},\{2\},\{1.2\}\}$
  \item $\mathscr{P}(\emptyset)=\{\emptyset\}$
  \item $\forall A: A\cap\emptyset=\emptyset$
  \item $\forall A:A\cup\emptyset=A$
  \item $\forall A,B,C: A\cap(B\cup C)=(A\cap B)\cup (B\cap C)$
  \item $\forall A,B,C: A\cup (B\cap C)=(A\cup B)\cap (A\cup C)$
  \item $\forall A,B,C: A\backslash(B\cup C)=(A\backslash B)\cap (A\backslash C)$
  \item $\forall A,B,C: A\backslash(B\cap C)=(A\backslash B)\cap (A\backslash C)$
\end{itemize}

\end{frame}
  % \begin{block}{Cryptographic System}
  %   An encryption is a function $enc:K\times P\to C$ from the cartesian product of the set of keys $K$ and the set of plaintexts $P$ to the set cyphertexts $C$. A decryption is a function $dec: K\times C\to P$. A suitable decryption has a key $k'$ such that \[\forall k\in K\forall x\in P: dec(k',enc(k,x))=x\]
  %   Alternative notation for $enc(k,m)$ is $enc_k(m)$.
  %   \end{block}
% \tikzset{rechteck/.style={rectangle, minimum
% width=20mm, minimum height=8mm,text centered, draw=black, fill=gray!15, line
% width=0.8}}
% \tikzset{rechteckfokus/.style={rectangle, minimum
% width=20mm, minimum height=8mm,text centered, draw=black, fill=green!15, line
% width=0.8}}
% \tikzset{rechteckcategory/.style={rectangle, minimum
% width=20mm, minimum height=8mm,text centered, draw=black, fill=gray!50, line
% width=0.8}}
% \begin{center}
% \begin{tikzpicture}[node distance=2cm] 

% \node (so) [rechteckcategory]{Security Objective};
% \node (dummy) [right of=so]{};
% \node (int) [rechteckfokus, right of=dummy]{Integrity};
% \node (conf) [rechteckfokus, above of=int]{Confidentiality};
% \node (ava) [rechteck, below of=int] {Availability};


% \draw (so) -- (int);
% \draw (so) -- (conf);
% \draw (so) -- (ava);


% \end{tikzpicture}
% \end{center}

% \begin{tikzpicture}[node distance=1.5cm] 


% \node (target) [rechteckfokus] {Cryptography Target};
% \node (dummy) [right of=target] {};
% \node (dummy2) [right of=dummy] {};
% \node (notarget) [rechteck, right of=dummy2] {No Cryptography Target};




% \end{tikzpicture}

% \tikzset{rechteck/.style={rectangle, minimum
% width=20mm, minimum height=8mm,text centered, draw=black, fill=gray!15, line
% width=0.8}}
% \tikzset{rechteckfokus/.style={rectangle, minimum
% width=20mm, minimum height=8mm,text centered, draw=black, fill=green!15, line
% width=0.8}}
% \tikzset{rechteckcategory/.style={rectangle, minimum
% width=20mm, minimum height=8mm,text centered, draw=black, fill=gray!50, line
% width=0.8}}
% \begin{center}
% \begin{tikzpicture}[node distance=4cm] 

% \node(pt)[rechteck] {Plaintexts};
% \node(dummy)[right of=pt] {};
% \node(ct)[rechteck, right of=dummy] {Ciphertexts};

% \draw[->] (pt) -- node [above] {$f$ easy to compute} (ct);
% \draw[->] (ct) -- node [below] {$f^{-1}$ hard to compute} (pt);
% \draw[->] (ct) edge[bend left] node [left, below] {$f^{-1}$ with trapdoor information easy to compute} (pt);

% % \draw[->] (-5.5,0) -- (5.5,0) node [below] {$\mathbb{R}$}; \foreach \x in {-5,...,5}
% % \draw (\x, 0.1) -- (\x, -0.1) node [below] {\x};


% \end{tikzpicture}
% \end{center}


% \begin{center}
% \begin{tikzpicture}[node distance=1cm]
%  \node (1)                  {$1$};
%  \node (5)  [above of=1]   {$5$};
%  \node (3)  [left of=5]  {$3$};
%  \node (2)  [left of=3]   {$2$};
%  \node (cont1) [right of=5]  {...};
%  \node (cont2)  [above of=5]  {...};
%  \node (15) [left of=cont2]  {$15$};

%  \node (6) [left of=15] {$6$};
%  \node (17) [right of=cont1] {$17$};
%  \node (cont3) [right of=17] {...};
%  \node (cont4) [left of=6] {...};
%  \node (cont5) [above of=15] {...};
%  %\node (cont6) [above of =cont5] {...};
%  \node (cont7) [above of =cont2] {...};
%  \node (cont8) [above of =17] {...};
%  \node (0)  [above of=cont7] {$0$};
%  \node (90) [left of=cont5] {$90$};
%   \node (30) [left of=90] {$30$};
%  \draw (6) -- (90);
%  \draw (15) -- (90);
%  \draw (2) -- (30);
%  \draw (15) -- (30);
%  \draw (15) -- (cont5);
%  \draw (1) -- (cont3);
%  \draw (1) -- (17);
%  \draw (2) -- (6);
%  \draw (3) -- (6);
%  \draw (1)   -- (5);
%  \draw (1)   -- (2);
%  \draw (1)   -- (3);
%  \draw (1)  -- (cont1);
%  \draw (cont1)  -- (cont2);

%  \draw (cont7)  -- (0);
%  \draw (3)   -- (15);
%  \draw (5)   -- (15);
 
% \end{tikzpicture}
% \end{center}

\end{document}
\documentclass{beamer}
\usetheme{CambridgeUS}
\usepackage{tikz}
\usetikzlibrary{matrix,arrows,fit,positioning, mindmap, trees}

\usepackage[latin1]{inputenc}
\usefonttheme{professionalfonts}
\usepackage{times}
\usepackage{xmpmulti}
\usepackage{animate}
\usepackage{amsmath}
\usepackage{verbatim}
%\usetheme{Boadilla}
%\usecolortheme{crane}
\title{Theoretische Grundlagen der Informatik}

\author{Prof.\ Dr.-Ing. Sebastian Schlesinger}
%\institute{Berlin School of Economics and Law}
\date{\today}
\begin{document}
 \begin{frame}
\titlepage
\end{frame}
\begin{frame}
\frametitle{Lernziele dieser Vorlesung}
Verst\"andnis f\"ur 
\begin{itemize}
\item grundlegende mathematische Notation
\item das F\"uhren mathematischer Beweise
\end{itemize}
\end{frame}
%\begin{frame}
% \frametitle{Outline}
% \tableofcontents
% \end{frame}
%\section{Notationen}
%\subsection{Symmetric va. asymmetric ciphers}
 
\begin{frame}
\frametitle{Mengen}

Test


\end{frame}


% \tikzset{rechteck/.style={rectangle, minimum
% width=20mm, minimum height=8mm,text centered, draw=black, fill=gray!15, line
% width=0.8}}
% \tikzset{rechteckfokus/.style={rectangle, minimum
% width=20mm, minimum height=8mm,text centered, draw=black, fill=green!15, line
% width=0.8}}
% \tikzset{rechteckcategory/.style={rectangle, minimum
% width=20mm, minimum height=8mm,text centered, draw=black, fill=gray!50, line
% width=0.8}}
% \begin{center}
% \begin{tikzpicture}[node distance=2cm] 

% \node (so) [rechteckcategory]{Security Objective};
% \node (dummy) [right of=so]{};
% \node (int) [rechteckfokus, right of=dummy]{Integrity};
% \node (conf) [rechteckfokus, above of=int]{Confidentiality};
% \node (ava) [rechteck, below of=int] {Availability};


% \draw (so) -- (int);
% \draw (so) -- (conf);
% \draw (so) -- (ava);


% \end{tikzpicture}
% \end{center}

% \begin{tikzpicture}[node distance=1.5cm] 


% \node (target) [rechteckfokus] {Cryptography Target};
% \node (dummy) [right of=target] {};
% \node (dummy2) [right of=dummy] {};
% \node (notarget) [rechteck, right of=dummy2] {No Cryptography Target};




% \end{tikzpicture}

% \tikzset{rechteck/.style={rectangle, minimum
% width=20mm, minimum height=8mm,text centered, draw=black, fill=gray!15, line
% width=0.8}}
% \tikzset{rechteckfokus/.style={rectangle, minimum
% width=20mm, minimum height=8mm,text centered, draw=black, fill=green!15, line
% width=0.8}}
% \tikzset{rechteckcategory/.style={rectangle, minimum
% width=20mm, minimum height=8mm,text centered, draw=black, fill=gray!50, line
% width=0.8}}
% \begin{center}
% \begin{tikzpicture}[node distance=4cm] 

% \node(pt)[rechteck] {Plaintexts};
% \node(dummy)[right of=pt] {};
% \node(ct)[rechteck, right of=dummy] {Ciphertexts};

% \draw[->] (pt) -- node [above] {$f$ easy to compute} (ct);
% \draw[->] (ct) -- node [below] {$f^{-1}$ hard to compute} (pt);
% \draw[->] (ct) edge[bend left] node [left, below] {$f^{-1}$ with trapdoor information easy to compute} (pt);

% % \draw[->] (-5.5,0) -- (5.5,0) node [below] {$\mathbb{R}$}; \foreach \x in {-5,...,5}
% % \draw (\x, 0.1) -- (\x, -0.1) node [below] {\x};


% \end{tikzpicture}
% \end{center}


% \begin{center}
% \begin{tikzpicture}[node distance=1cm]
%  \node (1)                  {$1$};
%  \node (5)  [above of=1]   {$5$};
%  \node (3)  [left of=5]  {$3$};
%  \node (2)  [left of=3]   {$2$};
%  \node (cont1) [right of=5]  {...};
%  \node (cont2)  [above of=5]  {...};
%  \node (15) [left of=cont2]  {$15$};

%  \node (6) [left of=15] {$6$};
%  \node (17) [right of=cont1] {$17$};
%  \node (cont3) [right of=17] {...};
%  \node (cont4) [left of=6] {...};
%  \node (cont5) [above of=15] {...};
%  %\node (cont6) [above of =cont5] {...};
%  \node (cont7) [above of =cont2] {...};
%  \node (cont8) [above of =17] {...};
%  \node (0)  [above of=cont7] {$0$};
%  \node (90) [left of=cont5] {$90$};
%   \node (30) [left of=90] {$30$};
%  \draw (6) -- (90);
%  \draw (15) -- (90);
%  \draw (2) -- (30);
%  \draw (15) -- (30);
%  \draw (15) -- (cont5);
%  \draw (1) -- (cont3);
%  \draw (1) -- (17);
%  \draw (2) -- (6);
%  \draw (3) -- (6);
%  \draw (1)   -- (5);
%  \draw (1)   -- (2);
%  \draw (1)   -- (3);
%  \draw (1)  -- (cont1);
%  \draw (cont1)  -- (cont2);

%  \draw (cont7)  -- (0);
%  \draw (3)   -- (15);
%  \draw (5)   -- (15);
 
% \end{tikzpicture}
% \end{center}

\end{document}